\documentclass[11pt]{article}
\usepackage[a4paper,width=160mm,top=25mm,bottom=25mm]{geometry}
\usepackage{graphicx}
\usepackage{fancyhdr}
\usepackage[spanish, mexico]{babel}
\usepackage[utf8]{inputenc}
\usepackage{amsmath}
\usepackage{gensymb}
\usepackage{upgreek}
\usepackage{mathtools}
\usepackage{xfrac}
\usepackage{hyperref}
%\usepackage[
%    backend=biber,
%    style=apa,
%  ]{biblatex}
\usepackage{csquotes}
%\addbibresource{bibresource}

\setlength{\headheight}{15pt}
\pagestyle{fancy}
\fancyhf{}
\lhead{Edificios Sustentables}
\chead{Transferencia de calor}
\rhead{Rodrigo Muñoz Sánchez}
\rfoot{\thepage}
\renewcommand{\headrulewidth}{0pt}

\renewcommand{\theenumiii}{\roman{enumiii}}

\title{Ejercicios de transferencia de calor}
\author{Rodrigo Muñoz}
\date{Enero 2022}

\graphicspath{Images/}

\begin{document}

\maketitle

\section{}

Se tiene una casa cuadrada de $ 3 \times 3 \times 3 \left[ m \right] $, a nivel del mar, con una temperatura de $ 25 \left[ \degree C \right] $ y humedad relativa de $ 30\% $. La casa tiene una ventana de $ 1 \left[ m^2 \right] $ en caras opuestas. Si incide un viento de $ 3 \left[ \sfrac{ km }{ h } \right] $ perpendicular a la ventana, calcula el gasto de viento en el interior.

De acuerdo con la NTC-Viento-2017, los coeficientes de presión son:

\begin{itemize}
    \item
    Barlovento: $ C_{ v \, 1 } = 0.8 $
    
    \item
    Sotavento: $ C_{ v \, 2 } = -0.4 $

\end{itemize}

Las características del aire son:

\begin{itemize}
    \item
    $ \rho _a = 1.033 \left[ \sfrac{ kg }{ m^3 } \right] $
    
    \item
    $ v = 3 \left[ \sfrac{ km }{ h } \right] = 0.833 \left[ \sfrac{ m }{ s } \right] $

\end{itemize}

Calculamos la diferencia de presión:

\[ \Delta P_v = \frac{1}{2} \rho _a v^2 \left( C_{ v \, 2 } - C_{ v \, 1 } \right) = 0.5 \cdot 1.033 \cdot 0.833^2 \left[ 0.8 - \left( -0.4 \right) \right] = 0.43 \left[ Pa \right] \]

Como no hay más información acerca de las ventanas, tomamos:

\[ C_{ d \, 1 } = C_{ d \, 2 } = 0.65 \]

\[ \left( AC_d \right) _{ eff } = \frac{ 1 }{ \sqrt{ \sum{ \frac{ 1 }{ \left(  A_i C_{ d \, i } \right) ^2 } } } } = \frac{ 1 }{ \sqrt{ \frac{ 1 }{ \left( 1 \cdot 0.65 \right) ^2 } } } \]

\[ \left( AC_d \right) _{ eff } = 0.460 \left[ m^2 \right] \]

Por último, calculamos el gasto.

\[ Q = \left( AC_d \right) _{ eff } \sqrt{ \frac{ 2 \Delta P }{ \rho _a } } = 0.460 \sqrt{ \frac{ 2 \cdot 0.430 }{ 1.033 } } \]

\[ \boxed{ Q = 0.420 \left[ \sfrac{ m^3 }{ s } \right] } \]

Para trabajar con arreglos de ventanas más complejos, podemos hacer superposición de las presiones:

\section{}

Se tiene un edificio a nivel del mar con una ventana de $ 1 \left[ m^2 \right] $ a $ 1 \left[ m \right] $ de altura sobre el piso y otra ventana de la mitad de tamaño a $ 5 \left[ m \right] $. En el exterior se tiene una temperatura de $ 25 \left[ \degree C \right] $ y una humedad relativa de $ 50\% $, mientras que en el interior se tiene $ 35 \left[ \degree C \right] $ y $ R_h = 30\% $. Calcula el gasto de aire y la altura del eje neutro.


Las características del aire son:

\begin{itemize}
    \item
    $ \rho _{ ext } = 1.166 \left[ \sfrac{ kg }{ m } \right] $

    \item
    $ \rho _{ int } = 1.126 \left[ \sfrac{ kg }{ m } \right] $

\end{itemize}

La diferencia de presión por efecto chimenea será:

\[ \Delta P_c = \Delta P_e + \Delta P_s = \left( \rho _{ ext } - \rho _{ int } \right) g \Delta z \]

\[ \Delta P_c = \left( 1.166 - 1.126 \right) 9.81 \cdot 4 = 1.57 \left[ Pa \right] \]

Como no hay más información acerca de las ventanas, tomamos:

\[ C_{ d \, 1 } = C_{ d \, 2 } = 0.65 \]

\[ \left( AC_d \right) _{ eff } = \frac{ 1 }{ \sqrt{ \sum{ \frac{ 1 }{ \left(  A_i C_{ d \, i } \right) ^2 } } } } = \frac{ 1 }{ \sqrt{ \frac{ 1 }{ \left( 1 \cdot 0.65 \right) ^2 } + \frac{ 1 }{ \left( 0.5 \cdot 0.65 \right) ^2 } } } \]

\[ \left( AC_d \right) _{ eff } = 0.291 \left[ m^2 \right] \]

Calculamos el gasto.

\[ Q = \left( AC_d \right) _{ eff } \sqrt{ \frac{ 2 \Delta P }{ \rho _a } } = 0.291 \sqrt{ \frac{ 2 \cdot 1.570 }{ 1.166 } } \]

\[ Q = 0.478 \left[ \sfrac{ m^3 }{ s } \right] \]

Ahora, establecemos las ecuaciones de variación de presión respecto a la altura:

\[ P_{ int } = - \rho _{ int } g h = -11.04 h \]

\[ P_{ ext } = \Delta P_e - \rho_{ ext } g h = \Delta P_e - 11.45 h \]

Como solo hay una ventana de salida y una de entrada, $ Q = Q_e = Q_s $.

\[ \Delta P_e = \frac{ \rho _{ ext } Q_e^2 }{ 2 A_e^2 C_d^2 } = \frac{ 1.126 \cdot 0.478^2 }{ 2 \cdot 1^2 \cdot 0.65^2} = 0.343 \left[ Pa \right] \]

\[ P_{ ext } = 0.343 - 11.45 h \]

Por último, despejamos la altura del eje neutro:

\[ P_{ int } \left( h_n \right) - P_{ ext } \left( h_n \right) = 0 \]

\[ -11.04 h_n - \left( 0.343 -11.45 h_n \right) = 0 \]

\[ 0.41 h_n - 0.343 = 0 \]

\[ \boxed{ h_n 0.84 \left[ m \right] } \]

Esta altura se mide sobre el eje de la ventana inferior.

%nocite{*}
%\printbibliography

\end{document}