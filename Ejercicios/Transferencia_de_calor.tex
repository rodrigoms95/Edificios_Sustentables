\documentclass[11pt]{article}
\usepackage[a4paper,width=160mm,top=25mm,bottom=25mm]{geometry}
\usepackage{graphicx}
\usepackage{fancyhdr}
\usepackage[spanish, mexico]{babel}
\usepackage[utf8]{inputenc}
\usepackage{amsmath}
\usepackage{gensymb}
\usepackage{upgreek}
\usepackage{mathtools}
\usepackage{xfrac}
\usepackage{hyperref}
%\usepackage[
%    backend=biber,
%    style=apa,
%  ]{biblatex}
\usepackage{csquotes}
%\addbibresource{bibresource}

\setlength{\headheight}{15pt}
\pagestyle{fancy}
\fancyhf{}
\lhead{Edificios Sustentables}
\chead{Transferencia de calor}
\rhead{Rodrigo Muñoz Sánchez}
\rfoot{\thepage}
\renewcommand{\headrulewidth}{0pt}

\renewcommand{\theenumiii}{\roman{enumiii}}

\title{Ejercicios de transferencia de calor}
\author{Rodrigo Muñoz}
\date{Enero 2022}

\graphicspath{Images/}

\begin{document}

\maketitle

\section{Conducción - Muro simple}

Muro: concreto de 20 cm, 3 m\textsuperscript{2}
Temperatura: exterior 30\degree C, interior 20\degree C

Obtener la densidad de flujo de calor, el calor transmitido durante un periodo de 2 h, y graficar el perfil térmico.

\subsection{Paso 1: esquema general}

Positivo en este sentido.

\subsection{Paso 2: esquema de resistencias térmicas}

Se preserva el sentido.

\subsection{Paso 3: calcular las resistencias térmicas}

\[ e = 10 \left[ cm \right] = 0.2 \left[ m \right] \]

Conductividad térmica del concreto

\[ \lambda = 1.74 \left[ \sfrac{ W }{ m K } \right] \]

\[ R = \frac{ e }{ \lambda } = \frac{ 0.2 }{ 1.74 } = 0.1149 \left[ \sfrac{ m ^ 2 K }{ W } \right] \]

\subsection{Paso 4: calcular el flujo}

La temperatura se debe trabajar en Kelvins para que las unidades coincidan. Solo cuando se trabaja con diferencias de temperatura se pueden utilizar grados centígrados, pero al hacer despejes esto puede dar lugar a errores si no se tiene cuidado.

\[ \Delta T = T_e - T_i = \left( 30 + 273.15 \right) - \left( 20 + 273.15 \right) = 10 \left[ K \right] \]

\[ \varphi = \frac{ \Delta T }{ R } = \frac{ 10 }{ 0.1149 } = 87.03 \left[ \sfrac{ W }{ m ^ 2 } \right] \]

\[ \boxed{ \varphi = 87.03 \left[ \sfrac{ W }{ m ^ 2 } \right] } \]

El flujo va en el sentido escogido.

\[ Q = \varphi A \Delta T \]

$ \varphi $ es la densidad de flujo térmico. $ \Phi = \varphi A $ es el flujo térmico.

Análisis dimensional

\[ \left[ \frac{ W }{ m ^ 2 } \right] \left[ m ^ 2 \right] \left[ s \right] = \left[ W s \right] = \left[ J \right] \]

\[ Q = 87.03 \cdot 3 \cdot \left( 2 \cdot 3600 \right) = 1.88 \times 10 ^ 6 J  \]

Donde las 2 horas se convirtieron a segundos.

\[ Q = 1.88 MJ \text{ en } \Delta t = 2 \left[ h \right] \]

\subsection{Paso 5: perfil de temperatura}

El perfil de temperatura es una recta con pendiente $ \varphi R $ entre el inicio y el fin del muro. En el interior y el exterior es una recta horizontal de temperatura constante.

\section{Conducción - Muro compuesto}

Muro: concreto 20 cm
Aplanado: yeso 2 cm en exterior e interior
Temperatura: exterior 30\degree C, interior 20\degree C

Calcular la densidad de flujo térmico y graficar el perfil térmico.

\subsection{Paso 1: esquema general}

\subsection{Paso 2: esquema de resistencias térmicas}

En conducción, el flujo será constante a través de todas las capas.

Resistencia equivalente

El flujo de la resistencia equivalente es el mismo que en la resistencia completa.

\subsection{Paso 3: calcular las resistencias térmicas}

Resistencias individuales

Capas 1 y 3: aplanado de yeso
Capa 2: concreto

Si hay una capa de pintura, ésta suele ser muy delgada, por lo que no ofrece resistencia térmica alguna y no se considera.

\[ e_1 = e_3 = 0.03 \left[ m \right] \]

Conductividad del yeso

\[ \lambda _1 = \lambda _3 = 0.372 \left[ \sfrac{ W }{ m K } \right] \]

\[ R_1 = R_3 = 0.0538 \left[ \sfrac{ m ^ 2 K }{ W } \right] \]

\[ R_2 = 0.1149 \left[ \sfrac{ m ^ 2 K }{ W } \right] \]

Resistencia equivalente

\[ R_{ eq } = \sum{ R_i } = R_1 + R_2 + R_3 = 0.2225 \left[ \sfrac{ m ^ 2 K }{ W } \right] \]

\subsection{Paso 4: calcular el flujo}

\[ \varphi = \frac{ \Delta T }{ R_{ eq } } \]

\[ \boxed{ \varphi = 44.94 \left[ \sfrac{ W }{ m ^ 2 } \right] } \]

\subsection{Paso 5: perfil de temperatura}

Trabajando de la capa con mayor temperatura a la de menor temperatura:

\[ T_i = T_{ i - 1 } - \varphi R_i \]

\[ T_1 = T_{ ext } - \varphi R_1 = 27.6 \left[ \degree C \right] \]

\[ T_2 = T_{ 1 } - \varphi R_2 = 22.4 \left[ \degree C \right] \]

Finalmente, para comprobar:

\[ T_{ int } = T_{ 2 } - \varphi R_3 = 20 \left[ \degree C \right] \]

La variación de temperatura en cada capa es una recta con pendiente $ \varphi R_i $

El dibujo no está a escala.

\section{Muro con convección}

\subsection{Esquema general}

\subsection{Esquema de resistencias térmicas}

Puede haber varias capas de conducción, donde la resistencia intermedia por conducción se puede sustituir por varias resistencias, una por cada capa, o por una resistencia para toda la conducción, con valor de $ \sum{ \lambda ^{ -1 } _i } e_i $. En caso de representar todas las capas por separada, cada capa tendría una resistencia asociada y en la interfaz entre capa habría un nodo, con una temperatura $ T_i $ para cada interfaz. Representar todas las capas es importante para graficar el perfil de temperatura.

El flujo térmico es el mismo para todas las capas de conducción y de convección. Las capas de conducción y convección van en serie y no se puede considerar por separado la convección y la conducción.

Resistencia equivalente.

\subsection{Calcular las resistencias térmicas}

\textbf{Por conducción}

Cuando hay un muro compuesto

\[ R_{ muro } = \sum{ \lambda ^{ -1 } _i e_i } \]

\textbf{Por convección}

En el exterior:

\[ R_{ s \, ext } h_{ ext } ^{ -1 } \]

No olvidar el exponente negativo del coeficiente de convección.

Sin viento:

\[ R_{ s \, ext } = 0.13 \left[ \sfrac{ W }{ m ^ 2 K } \right] \]

Con viento:

\[ h_{ ext } = f \left( v \right) \]

En el interior:

\[ R_{ s \, int } h_{ int } ^{ -1 } \]

Usualmente no hay viento en el interior y por tanto sol hay un valor estándar de coeficiente de convección:

\[ R_{ s\, int } = 0.04 \left[ \sfrac{ W }{ m ^ 2 K } \right] \]

\textbf{Resistencia equivalente}

\[ R_{ eq } = \sum{ R_i } = R_{ s \, ext } + R_{ muro } + R_{ s \, int } \]

\subsection{Paso 4: calcular el flujo}

\[ \varphi = \frac{ \Delta T }{ R_{ eq } } \]

\subsection{Paso 5: perfil de temperatura}

\[ T_i = T_{ i - 1 } - \varphi R_i \]

donde $ T_i = T_{ ext }, T_{ s \, ext }, T_1, T_2, \ldots , T_n, T_{s \, int }, T_{ int } $

El perfil por con conducción sigue siendo una recta, pero el perfil por convección es una curva asíntotica desde la temperatura superficial del muro hacia la temperatura ya sea del interior o del exterior.

\subsection*{ Problema }

Se deja como ejericicio al lector calcular la densidad de flujo térmico y graficar el perfil de temperaturas para el siguiente caso:

Muro: tabique, con resistencia térmica de $ \lambda = 0.872 \left[ \sfrac{ W }{ m K } \right] $ y espesor de $ e = 7 \left[ cm \right]$

Temperatura exterior de 10 \degree C e interior de 20 \degree C.

\section{Radiación incidente solar en una ventana}

La radiación solar es radiación de onda corta.

\subsection{Esquema general}

$ \varphi _\tau $ corresponde a la radiación transmitida a través de la ventana hacia el interior del edificio, mientras que

\[ \varphi _\alpha = \alpha \varphi _\theta \]

corresponde a la porción absorbida por la ventana.

\subsection{Análisis de la radiación solar}

Para poder estudiar la radiación solar en la ventana, es necesario proyectar el vector de la radiación solar directa hacia el vector normal de la ventana. $ \alpha $ será el ángulo horizontal con respecto a la normal de la ventana, mientras que $ \beta $ el ángulo vertical. Es importante notar que el vector de la radiación solar se suele definir en función del acimut y la altura solar, por lo que hay que emplear la trigonometría para convertir a los ángulos definidos en función de la ventana.

\[ \varphi _\theta = \varphi _s \cos \alpha \cos \beta \]

Esta fórmula aplica para un muro vertical y cambia en el caso de un muro horizontal.

\subsection{Esquema de resistencias térmicas}

El esquema se puede aplicar fácilmente para un muro opaco con varias capas si se sustituye la resistencia del vidrio, $ R_v $, por una resistencia por conducción para el muro, $ R_{ muro } = \sum{ \lambda ^2 _{i} e_i } $, en cuyo caso, la radiación transmitida sería 0:

\[ \tau \varphi _\theta = 0 \]

Como ahora hay un nodo donde confluye la radiación solar, a lo largo de la ventana ahora habrá dos flujos, y habla que hacer el balance entre la entrada y salida de estos flujo sobre el nodo en cuestión. El flujo que normalmente buscamos despejar es $ \varphi _{ int } $, ya que es la densidad de flujo térmico que entra a la habitación, y por tanto corresponderá a las cargas sensibles a través de la pared en un análisis de ventilación mecánica y en un balance de calor.

El sentido de flujo sigue siendo arbitrario, pero siempre tiene que ser de mayor a menor temperatura, por lo que en este caso hemos supuesto:

\[ \begin{aligned} 
    & T_{ s \, ext } > T_{ ext } \\
    & T_{ s \, ext } > T_{ s int } > T_{ int }
\end{aligned} \]

\subsection{Ecuaciones de flujo}

\begin{equation}\label{eq:1}
    \varphi _{ ext } = \frac{ T_{ s \, ext } - T_{ ext } }{ R_{ ext } }
\end{equation}

\begin{equation}\label{eq:2}
    \varphi _{ int } = \frac{ T_{ s \, ext } - T_{ ext } }{ R_v + R_{ int } }
\end{equation}

\subsection{Balance de Energía en $ T_{ s \, ext } $}

\begin{equation}\label{eq:3}
    \alpha \varphi _\theta = \varphi _{ ext } + \varphi _{ int }
\end{equation}

\subsection{Deducción de fórmula general}

Ahora, despejamos $ T_{ s \, ext } $ de \ref{eq:1} y \ref{eq:2}

\begin{equation} \label{eq:4}
    T_{ s \, ext} = \varphi _{ ext } R_{ ext } + T_{ ext }
\end{equation}

\begin{equation} \label{eq:5}
    T_{ s \, ext } = \varphi \left( R_v + R_{ int } \right) + T_{ int }
\end{equation}

Igualamos \ref{eq:4} y \ref{eq:5} y despejamos $ \varphi _{ ext } $

\[ \varphi _{ ext } + R_{ ext } + T_{ ext } = \varphi \left( R_v + R_{ int } \right) + T_{ int } \]

\begin{equation}\label{eq:6}
    \varphi _{ ext } = \frac{ \varphi _{ int } \left( R_v + R_{ int } \right) + T_{ int } - T_{ ext } }{ R_{ ext } }
\end{equation}

Sustituimos \ref{eq:6} en \ref{eq:3} y despejamos $ \varphi _{ int } $

\[ \alpha \varphi _\theta = \frac{ \varphi _{ int } \left( R_v + R_{ int } \right) + T_{ int } - T_{ ext } }{ R_{ ext } } + \varphi_{ int } \]

\[ \alpha \varphi _ \theta R_{ ext } + T_{ ext } - T_{ int } = \varphi _{ int } + \left( R_v + R_{ int } + R_{ ext } \right) \]

\[  \boxed{ \varphi _{ int } = \frac{ T_{ ext } - T_{ int } + \alpha \varphi _\theta R_{ ext } }{ R_v + R_{ int } + R_{ ext } } } \]

Esta fórmula nos permite calcular $ \varphi _{ int } $ sin tener que calcular ni $ T_{ s \, ext }  $ ni $ T_{ s \, int } $.

\subsection*{Problema}

Calcular el flujo térmico y el perfil de temperatura asociado a una ventana con vidrio simple orientada hacia el este con las siguientes características:

\[ e = \sfrac{1}{4}" \]

\[ \lambda = 0.93 \left[ \sfrac{ W }{ m K } \right] \]

\[ T_{ ext } = 10 \left[ \degree C \right] \]

\[ T_{ ext } = 20 \left[ \degree C \right] \]

La radiación solar tiene un ángulo de 30\degree con respecto a la horizontal de 45\degree con respecto al sur, es decir, en dirección sureste. La intensida de la radiación solar es de:

\[ \varphi _s = 500 \left[ \sfrac{ W }{ m ^2 } \right] \]

\subsection{Ventanas dobles}

Las ventanas dobles tienen dos capas de vidrio con una capa de aire entre las dos, que se encuentra sellado contra el exterior, y a veces incluso está parcialmente evacuado o está lleno de un gas inerte con menor conductividad térmica que el aire. Como no hay movimiento de esta capa de aire, la capa interna solo transmitirá calor por conducción. En el exterior e interior sigue habiendo convección.

La radiación solar incide en el primer vidrio, donde una parte es absorbida y la otra transmitida. La porción transmitida incide en el segundo vidrio, donde también hay una porción absorbida. Debido a esto, ahora habrpa dos nodos en la resistencia térmica donde haya una confluencia de varios flujos, por lo que tendremos 2 ecuaciones de balance y tres flujos diferentes a lo largo de la ventana. Se encesita armar un sistema de ecuaciones de 5 ecuaciones y 5 incógnitas y despejar $ \varphi _{ int } $.


\section{Balance de calor}

Se tiene un paralelepípedo de $ 30 \times 30 \times 60 \left[ cm \right] $. Las paredes son de acero pintado de blanco con un espesor de $ e = 1 \left[ cm \right] $. La conductividad del acero es de $ \lambda = 52.3 \left[ \sfrac{ W }{ m K } \right] $. Todas las paredes están selladas en las orillas y esquinas con un chaflán de soldadura, por lo que se puede suponer que no hay infiltraciones y que los puentes térmicos son despreciables. Una de las caras de $ 30 \times 30 \left[ cm \right] $ recibe una radiación directa normal de $ \varphi _s = 100 \left[ \sfrac{ W }{ m ^2 } \right] $. La temperatura exterior es de 10 \degree C, hay un fuente interna de calor con un flujo de $ \Phi _{ fuente } = 100 \left[ W \right] $.

En este no hay que olvidar la diferencia entre la densidad de flujo térmico, que representa una potencia por unidad de área y se representa con la letra griega \emph{phi} minúscuala, $ \varphi \left[ \sfrac{ W }{ m ^2 } \right] $, y el flujo térmico, o de calor, que se representa con la letra griega \emph{phi} mayúscula y es la potencia total que atraviesa una pared, $ \Phi = \varphi A \left[ W \right] $

Se pide calcular la temperatura en el interior del paralelepípedo.

\subsection{Densidad de flujo a través de las paredes}

Todas las paredes tienen convección natural en interior y exterior más conducción.

\[ e_p = e_1 = e_2 = e_3 = e_4 = e_5 = e_6 = 0.01 \left[ m \right] \]

\[ \lambda _p = \lambda _1 = \lambda _2 = \lambda _3 = \lambda _4 = \lambda _5 = \lambda _6 = 52.3 \left[ \sfrac{ W }{ m K } \right] \]

\[ R_{ p \, 1 } = R_{ p \, 2 } = R_{ p \, 3 } = R_{ p \, 4 } = R_{ p \, 5 } = R_{ p \, 6 } = e_p \lambda _p = 1.912 \times 10 ^{ -4 } \left[ \sfrac{ m ^2 K }{ W } \right] \]

Para la pared de $ 30 \times 30 \left[ cm \right] $ sin radiación (pared 1) y las paredes de $ 30 \times 60 \left[ cm \right] $: 

Hemos puesto el sentido de flujo de tal manera que será positivo de afuera hacia adentro.

\[ R_{ s \, int \, i } = 0.04 \left[ \sfrac{ W }{ m ^2 K } \right] \]

Para la resistencia superficial exterior de las paredes horizontales, se toma en cuenta que el flujo es horizontal.

\[ R_{ s \, ext \, 1 } = R_{ s \, ext \, 3 } = R_{ s \, ext \, 4 } = R_{ s \, ext \, 6 } = 0.13 \left[ \sfrac{ W }{ m ^2 K } \right] \]

Para las paredes verticales, las resistencia estándar es diferete si el flujo es ascendente o si es descendente. Para la pared de arriba escogemos flujo ascendente y para la de abajo flujo descendente, lo que implica que estamos suponiendo que el flujo irá desde el interior del paralelepípedo hacia el exterior.

\[ R_{ s \, ext \, 2 } = 0.1 \left[ \sfrac{ W }{ m^2 K } \right] \]

\[ R_{ s \, ext \, 3 } = 0.17 \left[ \sfrac{ W }{ m^2 K } \right] \]

En caso de que la suposición que hemos hecho sea incorrecta, habrá que cambiar los valores de estas resistencias.

Ahora podemos calcular el flujo en cada pared.

Aunque tengan diferentes dimensiones, la densidad de flujo para todas las paredes será el mismo, ya que no depende del área de la pared.

\[ \varphi _{ int \, 1 } = \varphi _{ int \, 3 } = \varphi _{ \int 4 } = \frac{ T_{ a \, ext } - T_{ a \, int } }{ R_{ eq \, 1 } } = \frac{ 10 - T_{ a \, int } }{ 0.1702 } \]

donde

\[ R_{ eq \, 1 } = R_{ eq \, 3 } = R_{ eq \, 4 } = R_{ eq \, 6 } = R_{ p \, 1 } + R_{ s \, int \, 1 } + R_{ s \, ext \, 1 } = 0.1702 \left[ \sfrac{ W }{ m ^2 K } \right] \]

\[ \varphi _2 = \frac{ 10 - T_{ a \, int } }{ 0.1402 } \]

\[ \varphi _3 = \frac{ 10 - T_{ a \, int } }{ 0.2102 } \]

para la pared de $ 30 \times 30 \left[ cm \right] $ con radiación (pared 6):

\[ \varphi _6 = \frac{ T_{ a \, ext } - T_{ a \, int } }{ R_{ eq \, 2 } } = \frac{ 15.2 - T_{ ai } }{ 0.1702 } \]

\subsection{Balance de calor}

Sabemos que hay un estado de flujo estacionario, por lo que

\[  \sum{ \Phi _i } = 0 \longrightarrow \Phi _{ total } = 0 \]

Además, no hay puentes térmicos, por lo que:

\[ \Phi _{ pt } = 0 \]

donde

\[ \Phi _{ pt } = \sum{ \left( L_i \psi _i + \psi _{ pi } \right) }  \]

Como no hay ventliación ni infiltraciones:

\[ \Phi _v = 0 \]

Tampoco hay radiación que entre directamente hacia el interior:

\[ \Phi _{ \tau \, s } = 0 \]

donde

\[ \Phi _{ \tau \, s } = \sum{ A_i \tau _i \varphi _{ \theta \, i } } \]

y se da principalmente en ventanas.

El flujo total es:

\[ \Phi _{ total } = \Phi _{ pared } + \Phi _{ fuente } = 0 \]

donde

\[ \Phi _{ pared } = \sum{ A_i \, \varphi _{ int \, i } } = A_1 \, \varphi _{ int \, 1 } + A_2 \, \varphi _{ int \, 2 } + 2 A_1 \, \varphi _{ int \, 3 } + A_5 \, \varphi _{ int \, 5 } + A_6 \, \varphi _{ int \, 6 } + \]

\[ \Phi _{ pared } = 9 \left( \frac{ 10 - T_{ a \, int } }{ 0.1702 } \right) + 18 \left( \frac{ 10 - T_{ a \, int } }{ 0.1402 } \right) + 2 \cdot 18 \left( \frac{ 10 - T_{ a \, int } }{ 0.2102 } \right) + 9 \left( \frac{ 15.2 - T_{ a \, int } }{ 0.1702 } \right) \]

Haciendo álgebra, se obtiene:

\[ \Phi _{ pared } = 5,588 - 531 T_{ a \, int } \]

Sustituyendo en el balance de calor:

\[ \Phi _{ pared } + \Phi _{ fuente } = 0 \]

\[ \Phi _{ pared } + \Phi _{ fuente } = 5,588 -531 T_{ a \, int } + 100 \]

Despejamos la temperatura interior: 

\[ 5,688 - 531 T_{ a \, int } = 0 \]

\[ \boxed{
    T_{ a \, int } = 10.52 \left[ \degree C \right]
} \]

El flujo va de adentro hacia afuera, por lo que la hipótesis de convección en las paredes horizontales fue cierta.

\subsection*{Métodos numéricos}

En ocasiones no será fácil despejar $ T_{ a \, int} $, por lo que será necesario resolver mediante un método número. Una forma sencilla de atacar este problema es mediante el uso de la función OBJETIVO de Excel.

Tenemos el siguiente sistema de ecuaciones:

\[ T_{ a \, ext } \longrightarrow \text{valor conocido} \]

\[ T_{ a \, int } \longrightarrow \text{incógnita} \]

\[ \sum{ \Phi } = 0 \]

sin olvidar que:

\[ \Phi \left[ W \right] = \varphi A \]

\[ \sum{ \Phi } = \Phi _{ fuente } + \Phi _{ pared } \]

\[ \Phi _{ fuente } \longrightarrow \text{valor conocido} \]

\[ \Phi _{ pared } = f \left( T_{ a \, ext }, \, T_{ a \, int } \right) \]

Incluyendo ventanas, puertas, etc.

Las celdas de Excel se plantean de la siguiente manera:

\[ \begin{aligned}
    & \mathrm{A} 1 \rightarrow T_{ a \, ext } \longrightarrow \text{fijo} \\
    & \mathrm{A} 2 \rightarrow T_{ a \, int } \longrightarrow \text{establecer un valor inicial, por ejemplo 0 \degree C} \\
    & \mathrm{A} 3 \rightarrow \varphi _{ pared \, 1 } = f \left( T_{ a \, ext }, \, T_{ a \, int } \right) \longrightarrow  \left( \$ \mathrm{A} \$ 1, \, \$ \mathrm{A} \$ 2 \right) \\
    & \mathrm{A} 4 \rightarrow \varphi _{ pared \, 1 } = f \left( T_{ a \, ext }, \, T_{ a \, int } \right) \longrightarrow  \left( \$ \mathrm{A} \$ 1, \, \$ \mathrm{A} \$ 2 \right) \\
    & \vdots \\
    & \mathrm{A} 8 \rightarrow \varphi _{ pared \, 6 } = f \left( T_{ a \, ext }, \, T_{ a \, int } \right) \longrightarrow  \left( \$ \mathrm{A} \$ 1, \, \$ \mathrm{A} \$ 2 \right) \\
    & \vdots \\
    & \mathrm{A} 9 \rightarrow \Phi _{ fuente } \longrightarrow \text{fijo} \\
    & \\
    & \mathrm{A} 9 \;\;  \rightarrow A_1 \longrightarrow \text{fijo} \\
    & \mathrm{A} 10   \rightarrow A_2 \longrightarrow \text{fijo} \\
    & \mathrm{A} 11   \rightarrow A_3 \longrightarrow \text{fijo} \\
    & \vdots \\
    & \mathrm{A} 9 \rightarrow A_6 \longrightarrow \text{fijo} \\
    & \\
    &\mathrm{A} 10 \rightarrow \sum{ \Phi } = \Phi _{ fuente } + \Phi _{ pared } \longrightarrow f \left( \$ \mathrm{A} \$ 1, \, \ldots , \, \$ \mathrm{A} \$ 16 \right)
\end{aligned} \]

Como $ \mathrm{A} 1, \, \mathrm{A} 9, \, \ldots, \, \mathrm{A} 15 $ son valores fijos, y $ \mathrm{A} 3, \, \mathrm{A} 4, \, \mathrm{A} 8 $ son función únicamente de $ \mathrm{A} 2 $, entonces $ \mathrm{A} 16 $ es únicamente función de $ \mathrm{A} 2 $, por lo que en la función OBJETIVO buscaremos que la celda $ \mathrm{A} 6 $ sea 0, cambiando el valor de la celda $ \mathrm{A} 2 $, que es igual a $ T_{ a \, int } $ y es el resultado que buscamos.

\section{Radiación emitida por un muro}

Un muro se comporta casi como un cuerpo negro, y como todos los objetos, emite radiación térmica en forma de onda larga.

\subsection{Análisis de la radiación}

De la ley de Stefan-Boltzmann:

\[ \Phi _{ ij } = A_{ ij } F_{ ij } \sigma \varepsilon _i \varepsilon _j \left( T_i^4 - T_j^4 \right) \]

Es muy importante trabajar con la temperatura en Kelvins, ya que no se trata de una simple diferencia de temperaturas, por que éstas están elevadas a la cuarta potencia.

Recordando la relación entre el flujo y la densdidad de flujo

\[ \Phi = \varphi A \]

Llegamos a la fórmula de la emisión en onda larga.

\[ \varphi _e = F_o \sigma \varepsilon _{ s \, ext } \varepsilon _o \left( T_{ s\, ext }^4 - T_o^4 \right) \]

Donde el subíndice o indica el objeto contra el que la superficie de nuestro muro va a estar intercambiando radiación de onda larga.

El factor de forma entre la superficie y el objeto, $ F_o $ puede ser 1, entre una losa y el cielo en el caso simplificado, 0.5, entre un muro y el cielo o entre el muro y el suelo, o calcularse con las gráficas mencionadas con anterioridad.

\subsection{Esquema de resistencias térmicas}

Podemos llegar al esquema de resistencias térmicas partiendo del esquema para la radiación incidente y observando que será igual, pero con el sentido de la radiación invertido. Como la dirección que escogemos para la radiación es arbitraria, entonces esto solo implica un cambio de signo en la fórmula que representa ea éste esquema.

\[ \varphi _{ int } = \frac{ T_{ ext} - T_{ int } - \varphi _e R_{ ext } }{ R_m + R_{ int } + R_{ ext } } \]

\[ \varphi _{ int } = \frac{ T_{ ext} - T_{ int } - R_{ ext } F_o \sigma \varepsilon _{ s \, ext } \varepsilon _o \left( T_{ s\, ext }^4 - T_o^4 \right) }{ R_m + R_{ int } + R_{ ext } } \]

Como lo que nos interesa es tener $ \varphi _{ int } $ en función de $ T_{ ext } $ y $ T_{ int } $ hay que sustituir $ T_{ s \, ext } $ en la ecuación anterior:

\[ T_{ s \, ext } = T_{ ext } + \varphi _{ int } \left( R_m + R_{ int } \right) \]

\[ \boxed{ \varphi _{ int } = \frac{ T_{ ext} - T_{ int } - R_{ ext } F_o \sigma \varepsilon _{ s \, ext } \varepsilon _o \left\{ \left[ T_{ ext } + \varphi _{ int } \left( R_m + R_{ int } \right) \right] ^4 - T_o^4 \right\} }{ R_m + R_{ int } + R_{ ext } } } \]

\subsection*{Métodos numéricos}

La ecuación anterior tiene un problema, ya que no se puede despejar analíticamente. Para resolverla, es necesario entonces utilizar algún método númerico que mediante iteraciones nos permita resolver una ecuación de la forma:

\[ \varphi _{ int } = f \left( \varphi _{ int } \right) \]

Algunas de las opciones son las siguientes:

\begin{enumerate}
    \item
    \textbf{Newton Raphson}

    Es el método matemático que nos va a permitir tener la precisión que deseemos, pero puede tener un alto costo computacional. Además, requiere calcular una derivada que en ciertos casos puede tener una complejidad considerable. La ecuación anterior se debe rearreglar para tener la forma:

    \[ f \left( \varphi _{ int } \right) = 0 \]

    y se itera de la siguiente manera:

    \[ \varphi _{ int \, 0 } = 0 \]

    o algún otro valor arbitrario que consideremos cercano a la respuesta real.

    \[ \varphi _{ int \, i } = \varphi _{ int \, i - 1 } - \frac{ f \left( \varphi _{ int \, i - 1 } \right) }{ f' \left( \varphi _{ int \, i - 1 } \right) } \]

    la iteración continúa mientras que:

    \[ \Delta \varphi_i = \varphi _i - \varphi _{ i - 1 } > \delta \]

    donde $ \delta $ es la precisión requerida y suele tener un valor de alrededor de $ 1 \times 10 ^{ -8 } $. La presición llegar a lograrse muchas veces en 5 iteraciones.

    \item
    \textbf{Iteración simple}

    Una opción mucho más sencilla es simplemente tomar la forma generar de la ecuación para métodos numéricos e iterar sobre ella. El problema que esto puede tener es que la respuesta converge más lentamente, aunque rara vez es necesario hacer más de 10 iteraciones para lograr una precisión aceptable.

    \[ \begin{aligned}
        & \varphi _{ int \, 0 } = 0 \\
        & \varphi _{ int \, 1 } = f \left( \varphi _{ int \, 0 } \right) \\ 
        & \varphi _{ int \, 2 } = f \left( \varphi _{ int \, 1 } \right) \\ 
        & \vdots \\
        & \varphi _{ int \, 10 } = f \left( \varphi _{ int \, 1 } \right) \\ 
        & \varphi _{ int } \approx \varphi _{ int \, 10 }
    \end{aligned} \]

    \item
    \textbf{Excel: función OBJETIVO}

    De manera similar a la radiación incidente de onda corta, podemos resolver el problema de balance de calor contemplado la radiación de onda larga utilizado la función OBJETIVO de Excel, aplicando el método anterior de iteración simple a una hoja de cálculo:

    Tenemos el siguiente sistema de ecuaciones:

    \[ T_{ a \, ext } \longrightarrow \text{valor conocido} \]

    \[ T_{ a \, int } \longrightarrow \text{incógnita} \]

    \[ \sum{ \Phi } = 0 \]

    \[ \sum{ \Phi } = \Phi _{ fuente } + \Phi _{ pared } \]

    \[ \Phi _{ fuente } \longrightarrow \text{valor conocido} \]

    \[ \Phi _{ pared } = f \left( T_{ a \, ext }, \, T_{ a \, int } \right) \]

    Las celdas de Excel se plantean de la siguiente manera:

    \[ \begin{aligned}
        & \mathrm{A} 1 \rightarrow T_{ a \, ext } \longrightarrow \text{fijo} \\
        & \mathrm{A} 2 \rightarrow T_{ a \, int } \longrightarrow \text{establecer un valor inicial, por ejemplo 0 \degree C} \\
        & \\
        & \mathrm{D} 1 \rightarrow \varphi _{ int \, 0 } = 0 \\
        & \mathrm{D} 2 \rightarrow \varphi _{ int \, 1 } = f \left( \varphi _{ int \, 0 } \right) \\ 
        & \mathrm{D} 3 \rightarrow \varphi _{ int \, 2 } = f \left( \varphi _{ int \, 1 } \right) \\ 
        & \vdots \\
        & \mathrm{D} 11 \rightarrow \varphi _{ int \, 10 } = f \left( \varphi _{ int \, 1 } \right) \\ 
        & \\
        & \mathrm{E} 1 \ldots \mathrm{D} 11 \rightarrow \varphi _{ pared \, 1 } \\ 
        & \mathrm{F} 1 \ldots \mathrm{E} 11 \rightarrow \varphi _{ pared \, 2 } \\
        & \vdots \\
        & \mathrm{I} 1 \ldots \mathrm{I} 11 \rightarrow \varphi _{ pared \, 6 } \\
        & \vdots \\
        & \mathrm{A} 3 \rightarrow \Phi _{ fuente } \longrightarrow \text{fijo} \\
        & \mathrm{A} 4 \rightarrow A_1 \longrightarrow \text{fijo} \\
        & \mathrm{A} 5 \rightarrow A_2 \longrightarrow \text{fijo} \\
        & \vdots \\
        & \mathrm{A} 9 \rightarrow A_6 \longrightarrow \text{fijo} \\
        &\mathrm{A} 10 \rightarrow \sum{ \Phi } = \Phi _{ fuente } + \Phi _{ pared } \longrightarrow f \left( \$ \mathrm{A} \$ 3, \, \ldots , \, \$ \mathrm{A} \$ 8, \$ \mathrm{D} \$ 11, \, \ldots , \, \$ \mathrm{I} \$ 11 \right)
    \end{aligned} \]

    En la función OBJETIVO buscaremos que la celda $ \mathrm{A} 10 $ sea 0, cambiando el valor de la celda $ \mathrm{A} 2 $, que es igual a $ T_{ a \, int } $ y es el resultado que buscamos.  

    \[ \]

\end{enumerate}

\subsection*{Emisión a varias superficies}

La superficie del muro intercambiará radiación con todas las superficie que sean visibles para ella. Podemos convertir la fórmula estándar de emisión, que solo considera una superficie, a una fórmula que considere la emisión a varias superficies.

\[ \varphi _e = \sum{ \varphi _{ e \, i } } = \sigma \varepsilon _{ s \, ext } \sum{ F_i \varepsilon _i \left( T_{ s \, ext }^4 - T_i^4 \right) } \]

Que desarrollado queda como:

\[ \boxed{ \varphi _e = \sigma \varepsilon _{ s \, ext } \sum{ F_i \varepsilon _i \left\{ \left[ T_{ a \, int } + \varphi _{ int } \left( R_{ int } + R_m \right) \right] ^4 - T_i^4 \right\} } } \]

\subsection{Radiación total}

Durante el día, un muro expuesto al sol va a tener tanto radiación solar incidente directa en onda corta como emisión de radiación en onda larga. Podemos generar entonces una función general para la transferencia de calor en un muro con radiación en el exterior:

\[ \varphi _\theta = \varphi _s \cos \alpha \cos \beta \]

\[ \varphi _r = \alpha \varphi _\theta - \varphi _e \]

\[ \boxed{ \varphi _{ int } = \frac{ T_{ ext} - T_{ int } - R_{ ext } \left\{ F_o \sigma \varepsilon _{ s \, ext } \sum{ F_i \varepsilon _i \left[ \left( T_{ a \, int } + \varphi _{ int } \left( R_{ int } + R_m \right) \right) ^4 - T_i^4 \right] } \right\} }{ R_m + R_{ int } + R_{ ext } } } \]

o de manera abreviada: 

\[ \boxed{ \varphi _{ int } = \frac{ T_{ ext} - T_{ int } - R_{ ext } \varphi _r }{ R_m + R_{ int } + R_{ ext } } } \]

\section{Transmisión en una fachada doble con ventilación}

Este ejemplo es muy parecido al de una ventana doble, con la excepción de que el aire en el interior no se encuentra estancado, si no que está en movimiento, y por ende, transporta calor y transmite al resto de las capas por convección.

El esquema es de aplicación amplia, por que las paredes pueden ser transparentes, en cuyo caso tendríamos una fachada acristalada doble con ventilación; la pared interior podría ser opaca, que sería el caso de un muro Trombe; o ambas paredes podrían ser opacas, que es el caso de una fachada ventilada. Estos ejemplos específicos serán mencionados más adelante en los capítulos de envolvente térmica y de control térmico natural.

\subsection{Esquema general}

\subsection{Análisis de la radiación solar}

\[ \varphi _\theta = \varphi _s \cos \alpha \cos \beta \]

\[ \varphi _r = \alpha \varphi _\theta - \varphi _e \]

\[ \varphi _{ \tau \, 1 } = \tau _1 \varphi _\theta \longrightarrow 0 \text{ si el primer muro es opaco} \]

\[ \varphi _{ \tau \, 2 } = \tau _2 \varphi _{ \tau 1 } = \tau _1 \tau _2 \varphi _\theta \longrightarrow 0 \text{ si el segundo muro es opaco} \]

\subsection{Esquema de resistencias térmicas}

\subsection{Flujo térmico del aire}

Como no hay acumulación de masa dentro de la fachada doble:

\[ \dot{m} _e = \dot{m} _s = \dot{m} _a \]

Del balance de calor:

\[ \Phi _v = \dot{m} c \Delta T \]

Convirtiendo a densidad de flujo de calor:

\[ \varphi _a = \frac{ \dot{m} _a c_a \left( T_e - T_a \right) }{ A } \]

Donde:

\begin{itemize}
    \item 
    $ c_a $: calor específico del aire

    \item
    $ A $: Área de la fachada

\end{itemize}

\subsection{Ecuaciones de flujo}

\[ \varphi _{ ext } = \frac{ T_{ a \, ext } - T_{ s \, ext \, 1 } }{ R_{ ext } } \]

\[ \varphi _1 = \frac{ T_{ s \, ext \, 1} - T_{ a } }{ R_{ v \, 1 } + R_{ a \, 1 } } \]

\[ \varphi _2 = \frac{ T_{ a } - T_{ s \, ext \, 2 } }{ R_{ a \, 2 } } \]

\[ \varphi _{ int } = \frac{ T_{ s \, ext \, 2 } - T_{ int } }{ R_{ v \, 2 } + R_{ int } } \]

\subsection{Balances de energía}

\[ \alpha _1 \varphi _\theta + \varphi _e = \varphi _1 \]

\[ \varphi _1 = \varphi _a + \varphi _2 \]

\[ \varphi _2 + \alpha _2 \tau _1 \varphi _\theta = \varphi _i \]

Tenemos ahora establecido un sistema que habrá que resolver.

%\nocite{*}
%\printbibliography

\end{document}